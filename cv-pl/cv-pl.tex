% Projekt: Nowoczesne CV. Część zestawu infographic-cv.
% Author : Michal Dyzma (c)
% Date   : 21.07.2015

\documentclass[11pt,a4paper,hidelinks]{article}

\usepackage[english,polish]{babel}

\usepackage{lipsum}
\usepackage{dashrule}
\usepackage{hyperref}
\usepackage{xcolor}
\usepackage{graphicx}
\usepackage{wasysym}
\usepackage{tikz}
\usepackage{tabularx,colortbl}

\usepackage{fontspec}
\usepackage{opensans}
\usepackage{fontawesome}

\setmainfont[]{opensans}
\newcommand\myfont[2]{\fontsize{#1}{#2}\selectfont}
\usetikzlibrary[arrows,snakes,backgrounds]

\begin{document}
	\thispagestyle{empty}
	
	
	\begin{tikzpicture}[remember picture,overlay,line width=0mm]
	
	% Naglowek
	\path [fill=black!80](current page.north west) rectangle (\paperwidth,1);
	
	% Imie i sentencja
	\node[xshift=2cm,yshift=3cm,text=white,font=\myfont{50}{48}]{\textsc{Michał Dyzma}};
	\node[xshift=2cm,yshift=1.6cm,text=white,font=\myfont{18}{48}]{True programmer has blood type 'C'};
	
	% Portale spolecznosciowe
	\node[yshift=12.25cm,xshift=8cm] at (current page.center)[text width=8cm,text=white,font=\myfont{25}{60},above]{\href{https://twitter.com/MichalDyzma}{\faTwitter} \hspace{.5cm} \href{https://www.facebook.com/profile.php?id=100005002682826}{\faFacebook} \hspace{.5cm} \href{https://www.linkedin.com/profile/view?id=316953832&trk=nav_responsive_tab_profile}{\faLinkedinSign} \hspace{.5cm} \href{https://github.com/mdyzma/infographic-cv}{\faGithub}};
	
	% Adres
	\node[xshift=-4cm,yshift=10cm] at (current page.center)[text width=8cm,text=black!80,align=left,font=\myfont{15}{60}]{\faHome \hspace{.2cm} Gen. Grochowskiego 10/28};
	\node[xshift=-4cm,yshift=9.5cm] at (current page.center)[text width=8cm,text=black!80,align=left,font=\myfont{15}{60}]{\hspace{.7cm} 05-500 Piaseczno};
	\node[xshift=5.5cm,yshift=10.5cm] at (current page.center)[text width=8cm,text=black!80,align=left,font=\myfont{15}{60}]{\faPhone \hspace{.2cm} 470 74 11 75};
	\node[xshift=5.5cm,yshift=9.8cm] at (current page.center)[text width=8cm,text=black!80,align=left,font=\myfont{15}{60}]{\faGlobe \hspace{.2cm} \href{http://michaldyzma.pl}{www.michaldyzma.pl}};
	\node[xshift=5.5cm,yshift=9.1cm] at (current page.center)[text width=8cm,text=black!80,align=left,font=\myfont{15}{60}]{\faEnvelope \hspace{.2cm} \href{mailto:mdyzma@gmail.com}{mdyzma@gmail.com}};
	
	%  Cienka kropkowana linia
	\draw [dotted,fill=black!80] (-3,-1.3) -- (14,-1.3);
	
	% Doswiadczenie
	%	\path [rotate=90,fill=gray!50] (-1.6,3.5) rectangle (-10,3.6);
	\node [xshift=-9.5cm,yshift=8.5cm,rotate=90]at (current page.center)[text width=8.5cm, font=\myfont{20}{32},text=black!80]{\textsc{\textbf{Doświadczenie}}};
	
	\node[xshift=-6cm,yshift=4.3cm] at (current page.center)[text width=8cm,text=black!80,align=left,font=\myfont{15}{15}]
	{
		\begin{tabularx}{0.95\paperwidth}{>{\raggedleft\scshape}p{4cm}X}
		\textbf{2014 - } & \textbf{Freelancer}\\
		& \textit{programista (python, C++, web-developer)}\\[0.4cm]
		\textbf{2014}  & \textbf{Międzynarodowy Instytut Biologii Molekularnej i Komórkowej}\\
		%& Lab. of Bioinformatics and Protein Engineering\\
		& \textit{biolog, programista}\\[0.3cm]
		\textbf{2009 - 2013}& \textbf{Instytut Podstawowych Problemów Techniki PAN,}\\
		%& Department of Mechanics and Physics of Fluids\\
		%& Laboratory of Modeling in Biology and Medicine\\
		& \textit{doktorant\\[0.3cm]
		\textbf{2006 - 2009}  & \textbf{Universit\'{e} Libre de Bruxelles}\\
		& H\^{o}pital Andr\'{e} Vesal\'{e}, Laboratoire de Sommeil\\
		& \textit{asystent naukowy}\\[0.3cm]
		\textbf{2004 - 2006}  & \textbf{Instytut Medycyny Pracyiom. Nofera w Łodzi}\\
		%& Department of Toxicology and Carcinogenesi\\
		& \textit{technik}\\
		\end{tabularx}
	};
	
	% Cienka kropkowana linia
	\draw [dotted,fill=black!80] (-3,-10.2) -- (14,-10.2);
	
	% Edukacja
	%	\path [rotate=90,fill=gray!50] (-10.4,3.5) rectangle (-14.7,3.6);
	\node [xshift=-9.6cm,yshift=-1.8cm,rotate=90]at (current page.center)[text width=5cm, font=\myfont{20}{32},text=black!80]{\textsc{\textbf{Edukacja}}};
	
	\node[xshift=-6cm,yshift=-2.6cm] at (current page.center)[text width=8cm,text=black!80,align=left,font=\myfont{15}{15}]
	{
		\begin{tabularx}{0.95\paperwidth}{>{\raggedleft\scshape}p{4cm}X}
		\textbf{2009 - 2013}& \textbf{Instytut Podstawowych Problemów Techniki PAN}\\
		& Laboratorium Modelowania w Biologii i Medycynie\\
		& \textit{studia doktoranckie}\\[0.3cm]
		\textbf{1998 - 2002} & \textbf{Uniwersytet Łódzki}\\
		& \textit{magister Neurobiologii, magister Genetyki}\\
		\end{tabularx}
	};
	
	% Cienka kropkowana linia
	\draw [dotted,fill=black!80] (-3,-14.9) -- (14,-14.9);
	
	%  Jezyki
	%	\path [rotate=90,fill=gray!50] (-15.3,3.5) rectangle (-19.5,3.6);
	\node [xshift=-9.51cm,yshift=-6.8cm,rotate=90]at (current page.center)[text width=5cm, font=\myfont{20}{32},text=black!80]{\textsc{\textbf{Języki}}};
	
	\node[xshift=-4cm,yshift=-7.5cm] at (current page.center)[text width=8cm,text=black!80,align=left,font=\myfont{17}{15}]
	{
		\begin{tabularx}{0.95\paperwidth}{>{\raggedright\scshape}p{3cm}X}
		\textbf{J. angielski}& \CIRCLE\CIRCLE\CIRCLE\CIRCLE\CIRCLE\Circle\\[0.4cm]
		\textbf{J. niemiecki} & \CIRCLE\CIRCLE\CIRCLE\Circle\Circle\Circle\\[0.4cm]
		\textbf{J. francuski} & \CIRCLE\Circle\Circle\Circle\Circle\Circle\\[0.4cm]
		\end{tabularx}
	};
	
	%Umiejetnosci 
	%	\path [rotate=90,fill=gray!50] (-15.3,-6.5) rectangle (-23.5,-6.6);		
	
	\node [xshift=.5cm,yshift=-8.2cm,rotate=90]at (current page.center)[text width=7.5cm, font=\myfont{20}{32},text=black!80]{\textsc{\textbf{Professional Skills}}};
	
	\node[xshift=5.5cm,yshift=-9.6cm] at (current page.center)[text width=8cm,text=black!80,align=left,font=\myfont{18}{18}]
	{
		\begin{tabularx}{0.95\paperwidth}{>{\raggedright\scshape}p{3cm}X}
		\textbf{Python}& \\[.2cm]
		\textbf{C++} & \\[.2cm]
		\textbf{HTML} & \\[.2cm]
		\textbf{CSS} & \\[.2cm]
		\textbf{PHP} & \\[.2cm]
		\textbf{\LaTeX} & \\[.2cm]
		\textbf{Matlab} & \\[.2cm]
		\textbf{Comsol} & \\[.2cm]
		\textbf{Linux} & \\[.2cm]
		\textbf{Windows} & \\[.2cm]
		\end{tabularx}
	};
	
	%Paski umiejetnosci
	
	% Python		
	\path [fill=black!80] (10.5,-15.3) rectangle (12.8,-15.85);
	\path [fill=gray!30]  (12.8,-15.3) rectangle (16,-15.85);
	% C++
	\path [fill=black!80] (10.5,-16.1) rectangle (13.2,-16.65);
	\path [fill=gray!30]  (13.2,-16.1) rectangle (16,-16.65);
	% HTML
	\path [fill=black!80] (10.5,-16.95) rectangle (13.8,-17.5);
	\path [fill=gray!30]  (13.8,-16.9) rectangle (16,-17.45);
	% CSS
	\path [fill=black!80] (10.5,-17.8) rectangle (13.2,-18.35);
	\path [fill=gray!30]  (13.2,-17.8) rectangle (16,-18.35);
	% PHP
	\path [fill=black!80] (10.5,-18.65) rectangle (11.5,-19.2);
	\path [fill=gray!30]  (11.5,-18.65) rectangle (16,-19.2);
	% LaTeX
	\path [fill=black!80] (10.5,-19.5) rectangle (14.6,-20.05);
	\path [fill=gray!30]  (14.6,-19.5) rectangle (16,-20.05);
	% Matlab
	\path [fill=black!80] (10.5,-20.35) rectangle (13.5,-20.9);
	\path [fill=gray!30]  (13.5,-20.35) rectangle (16,-20.9);
	% Comsol
	\path [fill=black!80] (10.5,-21.2) rectangle (11.8,-21.75);
	\path [fill=gray!30]  (11.8,-21.2) rectangle (16,-21.75);
	% Linux
	\path [fill=black!80] (10.5,-22.05) rectangle (13.1,-22.6);
	\path [fill=gray!30]  (13.1,-22.05) rectangle (16,-22.6);
	% Windows
	\path [fill=black!80] (10.5,-22.95) rectangle (13.8,-23.5);
	\path [fill=gray!30]  (13.8,-22.95) rectangle (16,-23.5);
	
	
	% Cienka kropkowana linia
	\draw [dotted,fill=black!80] (-3,-19.5) -- (5,-19.5);
	
	% Hobby
	%	\path [rotate=90,fill=gray!50] (-20,3.5) rectangle (-23.5,3.6);
	\node [xshift=-9.52cm,yshift=-10cm,rotate=90]at (current page.center)[text width=5cm, font=\myfont{20}{32},text=black!80]{\textsc{\textbf{Hobby}}};
	
	\node[xshift=-4.5cm,yshift=-11.5cm] at (current page.center)[text width=10cm,text=black!80,align=left,font=\myfont{15}{15}]
	{
		\begin{itemize}
		\item projekty DIY (n.p. RaspberryPi)
		\item \href{http://pl.spoj.com/status/mdyzma/}{\textbf{SPOJ.PL}}
		\item Rozwój serwisu \href{http://biokademia.pl}{\textbf{Bio::Kademia}}
		\end{itemize}
	};
	
	% Stopka
	\path [fill=black!80](current page.south west) rectangle (\paperwidth,-24);
	\end{tikzpicture}
	
	\newpage
	\thispagestyle{empty}
	
	\begin{tikzpicture}[remember picture,overlay,line width=0mm]
	\draw [fill=black!80](current page.south west) rectangle (\paperwidth,-24);
	\end{tikzpicture}
	
	
\end{document}